\ustcsetup{
  title              = {中国科学技术大学学位论文模板示例文档},
  title*             = {An example of thesis template for University of Science
                        and Technology of China \ustcthesisversion},
  author             = {李泽平},
  author*            = {Li Zeping},
  speciality         = {测控技术与仪器1105班},
  speciality*        = {Class 1105, Mathematics and Applied Mathematics},
  supervisor         = {刘文帅},
  supervisor*        = {Prof. XXX, Prof. XXX},
  % date               = {2017-05-01},  % 默认为今日
  % professional-type  = {专业学位类型},
  % professional-type* = {Professional degree type},
  department         = {机械工程学院},  % 院系,本科生需要填写
  department*        = {Mechanical Engineering Dept.},  % 院系,本科生需要填写
  student-id         = {2013060258},  % 学号,本科生需要填写
  %mydate               = {2023年6月20日}, 
  % secret-level       = {秘密},     % 绝密|机密|秘密|控阅,注释本行则公开
  % secret-level*      = {Secret},  % Top secret | Highly secret | Secret
  % secret-year        = {10},      % 保密/控阅期限
  %
  % 数学字体
  % math-style         = GB,  % 可选:GB, TeX, ISO
  math-font          = xits,  % 可选:stix, xits, libertinus
}


% 加载宏包

% 插图
\usepackage{graphicx}

% 三线表
\usepackage{booktabs}

% 跨页表格
\usepackage{longtable}

% SI 量和单位
\usepackage{siunitx}

% 参考文献使用 BibTeX + natbib 宏包
% 顺序编码制
\usepackage[sort]{natbib}
\bibliographystyle{ustcthesis-numerical}

% 著者-出版年制
% \usepackage{natbib}
% \bibliographystyle{ustcthesis-authoryear}

% 本科生参考文献的著录格式
% \usepackage[sort]{natbib}
% \bibliographystyle{ustcthesis-bachelor}

% 参考文献使用 BibLaTeX 宏包
% \usepackage[style=ustcthesis-numeric]{biblatex}
% \usepackage[bibstyle=ustcthesis-numeric,citestyle=ustcthesis-inline]{biblatex}
% \usepackage[style=ustcthesis-authoryear]{biblatex}
% \usepackage[style=ustcthesis-bachelor]{biblatex}
% 声明 BibLaTeX 的数据库
% \addbibresource{bib/ustc.bib}

% 配置图片的默认目录
\graphicspath{{figures/}}


% 用于写文档的命令
\DeclareRobustCommand\cs[1]{\texttt{\char`\\#1}}
\DeclareRobustCommand\pkg{\textsf}
\DeclareRobustCommand\file{\nolinkurl}

%%图、表的格式化引用
\newrefformat{fig}{\hyperref[#1]{图\ref*{#1}\,}}
\newrefformat{eq}{\hyperref[#1]{式\ref*{#1}\,}}
\newrefformat{tab}{\hyperref[#1]{表\ref*{#1}\,}}

%%listings设置
\usepackage{listings}
\usepackage{xcolor}
\lstset{
  tabsize=4, %
  frame=shadowbox, %把代码用带有阴影的框圈起来
  commentstyle=\color{red!50!green!50!blue!50},%浅灰色的注释
  rulesepcolor=\color{red!20!green!20!blue!20},%代码块边框为淡青色
  keywordstyle=\color{blue!90}\bfseries, %代码关键字的颜色为蓝色,粗体
  showstringspaces=false,%不显示代码字符串中间的空格标记
  stringstyle=\ttfamily, % 代码字符串的特殊格式
  keepspaces=true, %
  breakindent=22pt, %
  basicstyle=\footnotesize, %
  showspaces=false, %
  flexiblecolumns=true, %
  breaklines=true,%对过长的代码自动换行
  breakautoindent=true,%
  breakindent=4em,%
  texcl=true,
  aboveskip=1em %代码块边框
}

% hyperref 宏包在最后调用
\usepackage{hyperref}
